\section{Setup}

\subsection{Creating Your Repository}
Create a repository for your team by accepting the GitHub Classroom assignment, the link should have been posted on Ed.
The first member needs to create a team when accepting the assignment, where naming of the group is up to you.
The other member should join the team created by the first member.

Your repository is initially empty,
Import the project skeleton as follows:

\begin{minted}[tabsize=2]{bash}
  git clone git@github.com:EECS151-sp23/fpga_project_sp23-<your team name>.git
  cd fpga_project_sp23-<your team name>
  git remote add skeleton https://github.com/EECS150/fpga_project_sp23.git
  git pull skeleton main
  git push -u origin main
\end{minted}

To pull project updates from the skeleton repository (you will be notifed through Ed), run the following:
\begin{minted}[tabsize=2]{bash}
  git pull skeleton main
  git push
\end{minted}
while you can pull updates made by the other team member simply by \verb|git pull| after the other did \verb|git push|.
Resolve merge conflict carefully if it happens.

Remember to push your changes to the remote repository frequently.
\textbf{We are not responsible for any loss of your data.}
No extensions will be given for that reason.


\subsection{Integrating Designs from Labs}
You should copy some modules you designed from the labs, overwriting the provided skeleton files in \verb|hardware/src/io_circuits|.

\textbf{Files to copy from the labs:}
\begin{minted}{bash}
  debouncer.v
  synchronizer.v
  edge_detector.v
  uart_transmitter.v
\end{minted}


\subsection{Project Skeleton Overview}
\begin{itemize}
\item \verb|hardware|
  \begin{itemize}
  \item \verb|src|
    \begin{itemize}
    \item \verb|z1top.v|:
      Top level module. Your RISC-V CPU is instantiated here.
    \item \verb|z1top.xdc|:
      Constraint file for the top level module.
    \item \verb|EECS151.v|:
      The EECS151 register library. Some bugs have been fixed.
    \item \verb|clk_wiz.v|:
      Generates a clock signal for your CPU.
    \item \verb|io_circuits|:
      IO circuits from previous lab exercises.
    \item \verb|riscv_core/cpu.v|:
      Your RISC-V CPU design goes here.
      Some modules are provided.
      Don't change the names of provided modules and signals.
    \item \verb|riscv_core/opcode.vh|:
      Constant definitions for RISC-V opcodes and funct codes.
    \end{itemize}
  \item \verb|sim|
    \begin{itemize}
    \item \verb|asm_tb.v|:
      The testbench works with the software in \verb|software/asm|.
      The hex file will be written to the instruction/data memories.
    \item \verb|cpu_tb.v|:
      The testbench checks if your CPU can execute all the RV32I instructions (including CSR instructions) correctly,
      and can handle some simple hazards.
      This testbench directly edits the contents of the register file and the instruction/data memories for tests.
    \item \verb|isa_tb.v|:
      The testbench works with the RISC-V ISA test suite in \\ \verb|software/riscv-isa-tests|.
      We use 38 tests from the test suite.
      The testbench only runs one test at a time.
      To run all tests, run \verb|make isa-tests|.
      The hex file will be written to the instruction/data memories.
    \item \verb|c_tests_tb.v|:
      The testbench verifies the correct execution of the software in \verb|software/c_tests|.
      There are 6 tests provided.
      The hex file will be written to the instruction/data memories.
    \item \verb|uart_parse_tb.v|:
      The testbench works with the software in \verb|software/uart_parse|.
      It performs a simple write/read using the serial rx/tx lines.
      The hex file will be written to the instruction/data memories.
    \item \verb|echo_tb.v|:
      The testbench works with the software in \verb|software/echo|.
      The CPU reads a character sent from the serial rx line and echoes it back to the serial tx line.
      The hex file will be written to the instruction/data memories.
    \item \verb|mmio_counter_tb.v|:
      The testbench runs a small set of instructions and print out the memory mapped I/O counter values.
      This testbench directly edits the contents of the register file and the instruction/data memories for tests.
    \item \verb|bios_tb.v|:
      The testbench simulates the execution of the BIOS program in \\ \verb|software/bios|.
      It checks if your CPU can execute the instructions stored in the BIOS memory.
      The testbench also emulates user input sent over the serial rx line,
      and checks the BIOS message output obtained from the serial tx line.
    \item \verb|small_tb.v|:
      The testbench can be used to estimate the CPI of your design.
      It works with a smaller version of benchmark
      because the original benchmark is too large to do simulation.
      It reports the result checksum, cycle count and instruction count,
      and compares the checksum with a reference value.
      The program is loaded to the instruction/data memories without using BIOS,
      while the UART interface is
      used to get the result checksum and counter values.
    \item \verb|mem_path.vh|:
      Specifies the location of register file and memories.
      Some testbenches refer to this file to edit/check their values.
    \end{itemize}
  \item \verb|scripts|:
    \begin{itemize}
      \item \verb|run_all_sims.py|:
        Runs all testbenches above other than \\
        \verb|mmio_counter_tb.v|.
      \item \verb|hex_to_serial.py|:
        Sends a hex file to the FPGA through the serial rx line.
      \item \verb|run_fpga.py|:
        Calls \verb|hex_to_serial| and executes the program on the FPGA.
        It also prints out the result checksum and counter values.
      \item \verb|get_fmax.py|:
        Finds the CPU frequency from timing summary report.
      \item \verb|get_cost.py|:
        Calculates the cost of design from resource utilization report.
        The cost is a weighted sum of element counts.
        If your design uses elements that do not have cost assigned,
        please let us know.
      \item \verb|get_cpi.py|:
        Runs simulation or FPGA to get the geomean CPI for the benchmark programs.
        Simulation uses a smaller version of each benchmark,
        so CPI obtained in simulation is just an estimate.
        The final grade is based on the real CPI when running the original benchmarks on the FPGA.
      \item \verb|fom.py|:
        Calculates the figre of merits.
        By default, it reads the newest set of reports for cost and fmax,
        runs simulation for CPI, and displays the estimated FOM.
        Change parameters (commandline flags) to specify the report location or
        to calculate the real FOM through execution on the FPGA.
      \item \verb|*.tcl|:
        Scripts for Xilinx Vivado.
    \end{itemize}
  \item \verb|Makefile|:
    Makefile.
  \item \verb|README.md|:
    Explains make commands.
  \item \verb|stubs, sim_models|:
    Other modules from Xilinx library.
  \end{itemize}
\item \verb|software|
  \begin{itemize}
  \item \verb|Makefile, Makefrag|:
    Makefiles.
  \item \verb|151_library|:
    Files needed to compile software for our RISC-V CPU.
  \item \verb|asm|:
    Template of assembly tests.
    Modify this for a particular test you want to perform.
  \item \verb|riscv-isa-tests|:
    Tests from the RISC-V ISA test suite will be compiled here.
  \item \verb|c_tests|:
    Example C programs for tests.
    You may add a new one.
  \item \verb|uart_parse|:
    Simple tests using UART ports.
  \item \verb|echo|:
    The echo program, refer to the explanation for \verb|echo_tb.v| above.
  \item \verb|bios|:
    The BIOS program, which allows us to interact with our CPU via the UART.
  \item \verb|benchmark|:
    \begin{itemize}
    \item \verb|mmult|:
      This is a benchmark program to be run on the FPGA board.
      It generates 2 matrices and multiplies them.
      Then it returns a checksum to verify the correct result.
    \item \verb|bdd|:
      This is another benchmark program.
      It constructs binary decision diagrams for adder trees.
    \end{itemize}
  \item \verb|small|:
    This directory contains a smaller version of each benchmark.
  \end{itemize}
\item \verb|tips|
  \begin{itemize}
  \item \verb|fpga_checkout.md|:
    You may bring one FPGA to your home.
    Explains how to set up the environment and program the FPGA using Vivado in the instructional server.
  \end{itemize}
\item \verb|spec|:
  This specification document is located in this directory.
\item \verb|docs|:
  Documentation of your design goes to this directory.
\end{itemize}

\newpage

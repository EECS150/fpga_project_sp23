\section{Testing} \label{sec:testing}
\subsection{Unit Tests}
You should write unit tests for the isolated components of your CPU
such as the register file, decoder, and ALU in \verb|hardware/sim|.
The tests should contain assertions and check correct behavior
under several common and extreme conditions.

Run them just like you did in the labs.
For example, after you write \verb|sim/your_tb.v|, run \\
\verb|make sim/your_tb.fst| (iverilog) or \verb|make sim/your_tb.vpd| (VCS).
View the waveforms with \verb|gtkwave sim/your_tb.fst &| or \verb|dve -vpd sim/your_tb.vpd &|.

\subsection{Assembly Tests}
Once you connected the instruction memory in the instruction fetch stage and the data memory in the memory stage,
you should be able to run assembly tests using \verb|hardware/sim/asm_tb.v|.
Edit \verb|software/asm/start.s| by hand to add other tests you want.
It will be loaded into the instruction memory by the testbench,
where the \verb|RESET_PC| parameter is set to start the test in the IMEM instead of the BIOS.
Make sure you have used it in \verb|riscv_core/cpu.v|.

Initially, and if you change \verb|start.s|, you need to run \verb|make| in \verb|software|
before running simulation.

The \verb|start.s| contains the following instructions.
\begin{minted}{asm}
  # Test ADD
  li x10, 100         # Load argument 1 (rs1)
  li x11, 200         # Load argument 2 (rs2)
  add x1, x10, x11    # Execute the instruction being tested
  li x20, 1           # Set the flag register for the testbench
  # Now we check that x1 contains 300 in the testbench
\end{minted}
The \verb|asm_tb| toggles the clock one cycle at time and waits for register \verb|x20|
to be written with a particular value (in the above example: 1).
Once \verb|x20| contains 1, the testbench inspects the value in \verb|x1|
and checks if it is 300, which indicates your processor correctly executed the add instruction.

If the testbench times out it means \verb|x20| never became 1,
so the processor got stuck somewhere or \verb|x20| was written with another value.

You should add your own tests to verify that your processor can execute
different instructions correctly.
Modify and compile \verb|start.s| with your tests, add your checks to \verb|asm_tb.v|,
and then rerun the RTL simulation.

\subsection{CPU Test}
Once you are confident that your processor is working,
you will want to perform a comprehensive test.
The provided \verb|sim/cpu_tb.v| tests all the RV32I instructions.

To pass this testbench, you should have a working CPU implementation that can decode and execute
all the instructions in the spec, including the CSR instructions.
Several basic hazard cases are also tested.

Unlike other testbenches, this testbench does not work with any software code,
but rather it manually initializes the instructions and data in the memory blocks
as well as the register file content for each test.
The testbench does not cover reading from BIOS memory nor memory mapped IO.

\subsection{RISC-V ISA Tests}
These are tests developed in the RISC-V community.
To run the tests, run \verb|make isa-tests| or \verb|make sim/isa/lw.fst| for a specific test.

The simulation log details which tests passed and failed and the number of clock cycles elapsed.
If you're failing a test, debug using the generated assembly dump in
\verb|software/riscv-isa-tests|.

The assembly dump files are extremely helpful in debugging at this stage.
If you look into a particular dump file of a test (e.g., \verb|add.dump|),
it contains several subtests in series.
The CSR output from the simulation indicates which subtest is failing
to help you narrow down where the problem is, and you can start debugging from there.

\subsection{C Tests}
Next, you will test your processor with some small RISC-V C programs.
The C tests are in \verb|software/c_tests|.
You should go into each folder, understand what the program is trying to do.
The available tests are \verb|strcmp|, \verb|vecadd|, \verb|fib|, \verb|sum|,
\verb|replace|, and \verb|cachetest|.

To run the test, run \verb|make c-tests| or \verb|make sim/c_tests/fib.fst| for a specific program.
These tests could help reveal more hazard bugs in your implementation.
\verb|strcmp| is particularly important since it is frequently used in the BIOS program.

The tests use CSR instruction to indicate if they are passed
(e.g., write 1 to the CSR register if passed).
Following that practice, you can also write your custom C programs to further test your CPU.

As an additional tip for debugging, try changing the compiler optimization flag in the Makefile
of each software test (e.g., \verb|-O2| to \verb|-O1| or \verb|-O0|),
and see if your processor still passes the test.
Different compiler settings generate different sequences of assembly instructions,
and some might expose subtle hazard bugs yet to be covered by your implementation.

\subsection{UART and Echo Tests}
You should have your UART modules integrated with the CPU before running these tests.
These tests verify if your CPU is able to: check the UART status,
read a character from UART Receiver, and write a character to UART Transmitter.

Take a look at the software code \verb|software/echo/echo.c| to see what it does.
The testbench sends several characters over the serial rx line and sees if they are returned correctly.
The \verb|software/uart_parse/uart_parse.c| emulates the first few instructions of BIOS program.
The testbench checks if it sends and receives the correct sequence.

\subsection{MMIO Counter Test}
This test executes NOPs and a for loop with 10 iterations,
and prints out the values of counters for each.

Remember that we don't include NOPs in the instruction count,
and that only the completed instructions are counted
(the instructions that have passed the last stage of your pipeline).
The expected values for the instruction counts are around
2 (reset counters and load cycle count)
and 23 (loop initialization and 2 * 10 for loop iterations besides the previous two).

\subsection{BIOS Test}
We have provided a BIOS program in \verb|software/bios| that allows you to interact
with your CPU and download other programs over UART.
The BIOS program is an infinite loop that reads from the UART,
checks if the input string matches a known control sequence,
and then performs an associated action.
For detailed information on the BIOS, see Section \ref{sec:bios}.

The BIOS testbench \verb|sim/bios_tb.v| emulates the interaction between
the host and your CPU via the serial lines orchestrated by the BIOS program.
It tests four basic functions of the BIOS program:
\begin{itemize}
\item sending an invalid command
\item storing to an address (in IMEM or DMEM)
\item loading from an address (in IMEM or DMEM)
\item jumping to an address (from BIOSMEM to IMEM)
\end{itemize}


\newpage

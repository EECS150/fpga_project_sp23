\section{\blockDiagramTaskName: Block Diagram of RISC-V CPU}

This checkpoint is designed to guide the development of a three-stage pipelined RISC-V CPU
described in Section \ref{sec:spec}.

The second checkpoint will require significantly more time and effort than the first one.
As such, completing the first checkpoint (block diagram) in time for the design review is crucial to your success in this project.

Commit your block diagram and write-up to your team repository under \verb|docs|.


\subsection{Block Diagram}
The first checkpoint requires a detailed block diagram of your datapath.
The diagram should have a greater level of detail than a high level RISC datapath diagram.
You may complete this electronically or by hand,
but we strongly recommend writing it electrically.
You can use Adobe Illustrator, Inkscape, Google Drawings, draw.io or any program you want.
If working by hand, we recommend working in pencil and combining several sheets of paper for a larger workspace.

You should create a comprehensive and detailed design/schematic.
Enumerate all the control signals that you will need.
Be careful when designing the memory fetch stage
since all the memories we use (BIOS, instruction, data, IO) are synchronous.
You should be able to describe in detail any smaller sub-blocks in your diagram.

Although the diagrams from textbooks/lecture notes are a decent starting place,
remember that they often use asynchronous-read RAMs for the instruction and data memories,
and we will be using synchronous-read Block RAMs.
Therefore, we have one stage after IMEM,
one stage after DMEM (write-back to register file at the end of this stage),
and another stage in-between.
Note that the stages before IMEM (e.g. program counter) are not included in the stage count.
The reason behind is that Block RAMs in FPGA have a long clk-to-q delay and a small setup time.


\subsection{Questions}
Besides the block diagram, you will be asked to provide short answers to the following questions based on how you structure your block diagram.
Write up your answers in any format.
The questions are intended to make you consider all possible cases that might happen when your processor execute instructions, such as data or control hazards.
It might be a good idea to take a moment to think of the questions first, then draw your diagram to address them.

\begin{enumerate}
\item How many stages is the datapath you've drawn? (i.e. How many cycles does it take to execute one instruction?)
\item How do you handle ALU $\rightarrow$ ALU hazards?
  \begin{minted}{asm}
    addi x1, x2, 100 
    addi x2, x1, 100
  \end{minted}
\item How do you handle ALU $\rightarrow$ MEM hazards?
  \begin{minted}{asm}
    addi x1, x2, 100
    sw   x1, 0(x3)
    \end{minted}
\item How do you handle MEM $\rightarrow$ ALU hazards?
  \begin{minted}{asm}
    lw   x1, 0(x3)
    addi x1, x1, 100
  \end{minted}
\item How do you handle MEM $\rightarrow$ MEM hazards?
  \begin{minted}{asm}
    lw   x1, 0(x2)
    sw   x1, 4(x2)
  \end{minted}
  also consider:
  \begin{minted}{asm}
    lw   x1, 0(x2)
    sw   x3, 0(x1)
  \end{minted}
\item Do you need special handling for 2 cycle apart hazards?
  \begin{minted}{asm}
    addi x1, x2, 100
    nop
    addi x1, x1, 100
\end{minted}
\item How do you handle branch control hazards?
  (What prediction scheme are you using, or are you just injecting NOPs until the branch is resolved?
  If any prediction is done, what is the mispredict latency?
  What about data hazards in the branch?)
\item How do you handle jump control hazards?
  Consider jal and jalr separately.
\item What is the most likely critical path in your design?
\item Where do the UART modules, instruction, and cycle counters go?
  How are you going to drive \verb|uart_tx_data_in_valid| and \verb|uart_rx_data_out_ready|?
\item What is the role of the CSR register? Where does it go?
\item When do we read from the BIOS memory for instructions?
  When do we read from IMEM for instructions?
  How do we switch from BIOS address space to IMEM address space?
  In which case can we write to IMEM, and why do we need to write to IMEM?
  How do we know if a memory instruction is intended for DMEM or any IO device?
\end{enumerate}


\newpage

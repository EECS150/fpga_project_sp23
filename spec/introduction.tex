\section{Introduction}
The goal of this project is to familiarize EECS151/251A students with the methods and tools of digital design.
Working alone or in a team of two, you will design and implement a 3-stage pipelined RISC-V CPU with a UART for tethering.
After that, you will optimize your CPU to achieve a higher value of the form of merits.

You will use Verilog to implement this system, targeting the Xilinx PYNQ platform (a PYNQ-Z1 development board with a Zynq 7000-series FPGA).
The project will give you experience designing with RTL descriptions, resolving hazards in a simple pipeline, building interfaces, and teach you how to approach system-level optimization.

In tackling these challenges, your first step will be to map the high level specification to a design which can be translated into a hardware implementation.
After that, you will produce and debug that implementation.
These first steps can take significant time if you have not thought out your design prior to trying implementation.

As in previous semesters, your EECS151/251A project is probably the largest project you have faced so far here at Berkeley.
Good time management and good design organization is critical to your success.


\subsection{Tentative Deadlines}
The following is a brief description of each checkpoint.
Note that this schedule is tentative and is subjected to change as the semester progresses.

\begin{minipage}{\textwidth}
\vspace{2mm}
\begin{itemize}
\item \textbf{\blockDiagramDueDate \space - \blockDiagramTaskName} -
  Draw a schematic of your processor's datapath and pipeline stages,
  and provide a brief write-up of your answers to the questions.
  Also commit your design documents (block diagram + write-up) to \verb|docs|.
  {\it The deadline was pushed back because of spring break,
  but you should finish this checkpoint as soon as possible
  since \baseCPUTaskName \space is due \baseCPUDueDate.}

\item \textbf{\baseCPUDueDate \space - \baseCPUTaskName} -
  Implement a fully functional RISC-V processor core in Verilog.
  Your processor core should be able to run the \textbf{mmult} demo successfully.

\item \textbf{\finalCheckoffDueDate \space - Final Checkoff} -
  Processor optimization and checkoff.
    
\item \textbf{\finalReportDueDate \space - Project Report} -
  Project report due.
\end{itemize}
\vspace{2mm}
\end{minipage}


\clearpage
\subsection{Tentative Grading Rubric}
\begin{framed}
\begin{description}
\item[50\%] {\bf Functionality} at the final checkoff.
  You need to show your design passes all testbenches
  and executes \verb|mmult| correctly on the FPGA board.
\item[35\%] {\bf Optimization} at the final checkoff.
  The quality of your design will be evaluated according to the figure of merit.
  This score is contingent on implementing the correct functionality.
  A malfunctioning design will receive a zero in this category.
\item[5\%] {\bf Checkpoints}.
  The checkpoints are set to guide you to finish your design on time.
  To encourage you to follow the timing,
  checking off at each checkpoint on time makes up 5\% of your project grade in total.
  The weight of each checkpoint's score may vary.
\item[10\%] {\bf Project report}.
  The final report summarizing your project.
\end{description}
\end{framed}


\subsection{General Project Tips}
Document your project as you go.
You should comment your Verilog and keep your diagrams up to date.
Aside from the final project report (you will need to turn in a report documenting your project),
you can use your design documents to help the debugging process.

Finish the required features first.
Optimize your design after everything works well.
You should fully utilize the version control system (Git) to maintain a functioning design
while making changes incrementally.
\textbf{If your design does not work at the final checkoff, you will not get any credit for any optimization you did.}

\newpage

\section{Programming} \label{sec:programming}
\subsection{Manual Tests}
After programming, open screen to access the serial port:
\begin{minted}{bash}
  screen $SERIALTTY 115200
\end{minted}
Press the reset button (the right most button) to make the CPU PC go to the start of BIOS memory.

If all goes well, you should see a \verb|151 >| prompt after pressing return.
The following commands are available:
\begin{itemize}
\item \verb|jal <address>|: Jump to address (hex)
\item \verb|[sw, sb, sh] <data> <address>|: Store data (hex) to address (hex).
\item \verb|[lw, lbu, lhu] <address>|: Prints the data at the address (hex).
\item \verb|run|: Jump to \verb|0x10000000|.
\end{itemize}

As an example, running \verb|sw cafef00d 10000000| should write to the data memory
and running \verb|lw 10000000| should print the output \verb|10000000: cafef00d|.
Please also pay attention that writes to the instruction memory .
You may try \verb|sw ffffffff 20000000| that writes to the data memory,
where \verb|lw 10000000| still should yield \verb|cafef00d|.

Once you're done, make sure that you close screen using \verb|Ctrl-a| \verb|Shift-k|,
or other students won't be able to use the serial port!
If you can't access the serial port you can run \verb|killscreen| to kill all screen sessions.

\subsection{Loading Software Programs}
In addition to the command interface,
the BIOS allows you to load programs to the CPU.
With {\bf screen closed}, run:
\begin{minted}{bash}
  ./scripts/hex_to_serial.py <hex_file>
\end{minted}
This script stores the hex file to the subsequent addresses from \verb|0x30000000|,
which means data is written to both the data and instruction memories.
You may first try loading the echo program by
\begin{minted}{bash}
  ./scripts/hex_to_serial.py ../software/echo/echo.hex
\end{minted}

Then, jump to the loaded instructions in your screen session by \verb|run|.
It is just an alias of \verb|jal 10000000|.
In this case, as you loaded the echo program, you should see the characters you typed
sent back and printed on your screen.

After you make sure that your FPGA are executing the loaded program correctly,
reset your FPGA (the rightmost button) and try the mmult program
in \verb|../software/benchmark/mmult/mmult.hex|.
For running benchmark programs, you can use another script:
\begin{minted}{bash}
  ./script/run_fpga.py <hex_file>
\end{minted}
For \verb|mmult|, run \verb|./script/run_fpga.py ../software/benchmark/mmult/mmult.hex|.

This internally calls \verb|hex_to_serial.py| and issues \verb|run| command to begin execution.
It will display the results from the serial tx line to stdout if the program successfully finishes.

The \verb|mmult| program computes $S=AB$, where $A$ and $B$ are $64 \times 64$ matrices.
The program will print a checksum and the counters discussed in Memory Mapped IO.
The correct checksum is \verb|0001f800|.
If nothing printed out, there is likely a problem in your CPU with one of the instructions that is used by the BIOS but not mmult.

The program will also output the values of your instruction and cycle counters (in hex).
These can be used to calculate the CPI for this program.


\newpage
